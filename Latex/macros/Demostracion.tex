\usepackage[dvipsnames]{xcolor}
\usepackage{longtable}

\newenvironment{demostracion}[3]{
    \vspace{0.4cm}
    \noindent \textbf{#1 #2(#3)} \ensuremath{ \{}
    \begin{longtable}{p{0.02 \textwidth} p{0.95 \textwidth}}
}{
	\end{longtable}
	
    \noindent \ensuremath{\}}
    \vspace{0.15cm}
}

\newcommand{\estado}[1]{& \textsf{//Estado #1};\\}
\newcommand{\codigo}[1]{& \hspace*{-1em}\boldmath{\ensuremath{#1};}\\}
\newcommand{\vale}[1]{& \textsf{//vale }\ensuremath{#1};\\}
\newcommand{\valeN}[2]{& \textsf{//vale #1: }\ensuremath{#2};\\}
\newcommand{\implica}[1]{& \hangindent=4em \hangafter=+1 \textsf{//implica } \ensuremath{#1};\\}
\newcommand{\implicaN}[2]{& \hangindent=4em \hangafter=+1 \textsf{//implica #1: } \ensuremath{#2};\\}
\newcommand{\implicaC}[2]{& \hangindent=4em \hangafter=+1 \textsf{\color{#1} //implica } \ensuremath{\color{#1} #2};\\}
\newcommand{\justificacion}[1]{& \hangindent=3em \hangafter=0 \textsf{\color{Gray}//#1}\\}
\newcommand{\enumeracion}[3]{& \textsf{\color{#2}//(#1) #3}\\}
\newcommand{\invRep}[2]{\hangindent=4em \hangafter=1 \textsf{//invRep (#1):\newline} \ensuremath{#2};\\}
\newcommand{\abs}[2]{\hangindent=4em \hangafter=1 \textsf{//abs (#1):\newline} \ensuremath{#2};\\}