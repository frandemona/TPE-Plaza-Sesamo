\documentclass[10pt,a4paper,spanish]{article} 

\usepackage{a4wide}
\usepackage{amsmath, amscd, amsthm, latexsym}
\usepackage[spanish]{babel} % Le indicamos a LaTeX que vamos a escribir en espa�ol.
\usepackage[latin1]{inputenc} % Permite utilizar tildes y e�es normalmente
\usepackage{soul}
%\usepackage[utf8]{inputenc}
%\usepackage{bbm}
\usepackage{framed}
\usepackage{listings}
\usepackage{enumerate}
\usepackage{xspace}
\usepackage{longtable}
\usepackage{caratula}
\usepackage{fancyhdr}
\pagestyle{fancyplain}
%\overfullrule=2cm %Ver margenes en negro para overfull \hbox
\input{Algo1Macros}% Macros especificas para especificar problemas en AyEDI
%\usepackage{listings}
\usepackage{xcolor}

\definecolor{gray97}{gray}{.97}
\definecolor{gray75}{gray}{.75}
\definecolor{gray45}{gray}{.45}

\lstset{
	language=c++,
	frame=Ltb,
	framerule=0pt,
	aboveskip=0.5cm,
	framextopmargin=3pt,
	framexbottommargin=3pt,
	framexleftmargin=0.4cm,
	framesep=0pt,
	rulesep=.4pt,
	backgroundcolor=\color{gray97},
	rulesepcolor=\color{black},
	%
	stringstyle=\sffamily,
	showstringspaces = false,
	basicstyle=\small\sffamily,
	commentstyle=\color{gray45} \itshape,
	keywordstyle=\bfseries,
	%
	numbers=left,
	numbersep=15pt,
	numberstyle=\tiny,
	numberfirstline = false,
	breaklines=true
}
\renewcommand{\lstlistingname}{Archivo}

\lstnewenvironment{listing}[1][]
   {\lstset{#1}\pagebreak[0]}{\pagebreak[0]} Para el TPI

\newcommand{\comen}[2]{%
\begin{framed}
\noindent \textsf{#1:} #2
\end{framed}
}
% Aca solo vamos a poner el esqueleto del documento, pero no vamos a especificar nada.

% Encabezado
\lhead{Algoritmos y Estructuras de Datos I}
\rhead{Grupo 19 - Plaza Sesamo}
% Pie de pagina
\renewcommand{\footrulewidth}{0.4pt}
\lfoot{Facultad de Ciencias Exactas y Naturales}
\rfoot{Universidad de Buenos Aires}


\begin{document} % Todo lo que escribamos a partir de aca va a aparecer en el documento.

% Datos de caratula
\materia{Algoritmos y Estructura de Datos I}
\titulo{Trabajo Pr\'actico 1 - Especificaci\'on}
\subtitulo{Fat Food}
\grupo{Grupo 19 Plaza Sesamo}
\fechas{\today}
\integrante{Gagliardi, Hernan Gabriel}{810/12}{XXX@XXX.XXX}
\integrante{de Monasterio, Fransisco Jes\'us}{764/13}{XXX@XXX.XXX}
\integrante{Mart\'in Darricades, Mat\'ias Facundo}{480/13}{matiasamd@gmail.com}
\integrante{Solari Saban, Tom\'as Le\'on}{774/14}{tomassolari94@gmail.com}

% Pongan cuantos integrantes quieran

\maketitle


\newpage

% Para crear un indice
\tableofcontents
	

% Forzar salto de pagina
\clearpage


\newpage


\section{Tipos} %Defino los renombres de tipos b�sicos.

\input{tipos/tipos}

\section{Combo} %Defino el tipo combo, con su especificaci�n.

\begin{problema}{nuevoC}{b: Bebida, h: Hamburguesa, d: Energia}{Combo}
\requiere{ d \geq 0 \land d \leq 100} %tambien podria ser \requiere{energiaEnRango(d) }
\asegura{ bebida(\res) == b}
\asegura{ sandwich(\res) == h}
\asegura{ dificultad(\res) == d}
\end{problema}

\begin{problema}{bebidaC}{c: Combo}{Bebida}
\asegura{ \res == bebida(c)}
\end{problema}

\begin{problema}{sandwichC}{c: Combo}{Hamburguesa}
\asegura{ \res == sandwich(c)}
\end{problema}

\begin{problema}{dificultadC}{c: Combo}{Energia}
\asegura{ \res == dificultad(c)}
\end{problema}

 % Aca va la definici�n del tipo.

\begin{problema}{nuevoC}{b: Bebida, h: Hamburguesa, d: Energia}{Combo}
\requiere{ d \geq 0 \land d \leq 100} %tambien podria ser \requiere{energiaEnRango(d) }
\asegura{ bebida(\res) == b}
\asegura{ sandwich(\res) == h}
\asegura{ dificultad(\res) == d}
\end{problema}

\begin{problema}{bebidaC}{c: Combo}{Bebida}
\asegura{ \res == bebida(c)}
\end{problema}

\begin{problema}{sandwichC}{c: Combo}{Hamburguesa}
\asegura{ \res == sandwich(c)}
\end{problema}

\begin{problema}{dificultadC}{c: Combo}{Energia}
\asegura{ \res == dificultad(c)}
\end{problema}

 % La especificaci�n del tipo combo.

\newpage %Salto de p�gina


\section{Pedido} 


\begin{problema}{nuevoP}{n: \ent, e: Empleado, cs: [Combo]}{Pedido}
\end{problema}

\begin{problema}{numeroP}{p: Pedido}{\ent}
\end{problema}

\begin{problema}{atendioP}{p: Pedido}{Empleado}
\end{problema}

\begin{problema}{combosP}{p: Pedido}{[Combo]}
\end{problema}

\begin{problema}{agregarComboP}{p: Pedido, c: Combo}{}
\end{problema}

\begin{problema}{anularComboP}{p: Pedido, i:\ent}{}
\end{problema}

\begin{problema}{cambiarBebidaComboP}{p: Pedido, b: Bebida, i:\ent} {}
\end{problema}

\begin{problema}{elMezcladitoP}{p: Pedido}{}
\end{problema}



\begin{problema}{nuevoP}{n: \ent, e: Empleado, cs: [Combo]}{Pedido}
\end{problema}

\begin{problema}{numeroP}{p: Pedido}{\ent}
\end{problema}

\begin{problema}{atendioP}{p: Pedido}{Empleado}
\end{problema}

\begin{problema}{combosP}{p: Pedido}{[Combo]}
\end{problema}

\begin{problema}{agregarComboP}{p: Pedido, c: Combo}{}
\end{problema}

\begin{problema}{anularComboP}{p: Pedido, i:\ent}{}
\end{problema}

\begin{problema}{cambiarBebidaComboP}{p: Pedido, b: Bebida, i:\ent} {}
\end{problema}

\begin{problema}{elMezcladitoP}{p: Pedido}{}
\end{problema}


\newpage

\section{Local} 


\begin{problema}{stockBebidasL}{l: Local, b:Bebida}{Cantidad}
\requiere{ bebidaPerteneceLocal(b,l) } 
\asegura{ \res == stockBebidas(l,b)}
\end{problema}

\begin{problema}{stockSandwichesL}{l: Local, h:Hamburguesa}{Cantidad}
\requiere{ sandwichPerteneceLocal(h,l) } 
\asegura{ \res == stockSandwiches(l,h)}
\end{problema}

\begin{problema}{bebidasDelLocalL}{l: Local}{[Bebida]}
\asegura{ mismos (\res , bebidasDelLocal(l))}
\end{problema}
	
\begin{problema}{sandwichesDelLocalL}{l: Local}{[Hamburguesa]}
\asegura{ mismos (\res , sandwichesDelLocal(l))}
\end{problema}

\begin{problema}{empleadosL}{l: Local}{[Empleado]}
\asegura{ mismos (\res , empleados(l))}
\end{problema}

\begin{problema}{desempleadosL}{l: Local}{[Empleado]}
\asegura{ mismos (\res , desempleados(l))}
\end{problema}

\begin{problema}{energiaEmpleadoL}{l: Local, e:Empleado}{Energia}
\requiere{ empleadoPerteneceLocal(e,l) } 
\asegura{ \res == energiaEmpleado(l,e)}
\end{problema}

\begin{problema}{ventasL}{l: Local}{[Pedido]}
\asegura{ mismos (\res , ventas(l))}
\end{problema}

\begin{problema}{unaVentaCadaUno}{l:Local}{\bool}

\asegura{\res == lista}
%quienAtiendeVentasEmpleado(ventas(l),l) me da la lista de empleados actuales que atendieron los pedidos
\end{problema}

\begin{problema}{venderL}{l: Local, p:Pedido}{}
\requiere[numeroPedidoCorrecto]{numero(p) == mayorNumeroPedido(ventas(\pre{l})) + 1}
\requiere[empleadoDelLocal]{empleadoPerteneceLocal(atendio(p),\pre{l})}
\requiere[tieneEnergiaSuficiente]{energiaEmpleado(\pre{l},atendio(p)) \geq energiaPedido(p)} % No estoy seguro de este requiere
\requiere[stockSandwichSuficiente]{(\forall \selector{h}{sandwichDistintos(combos(p))}) ( stockSandwiches(\pre{l}, h) \\ - cuentaSandwich(h,combos(p))) > 0}
\requiere[stockBebidaSuficiente]{(\forall \selector{b}{bebidaDistintos(combos(p))}) ( stockBebidas(\pre{l}, b) \\ - cuentaBebida(b,combos(p))) > 0}
\modifica{l}
\asegura{mismos(bebidasDelLocal(l),bebidasdelLocal(\pre{l}))}
\asegura{mismos(sandwichesDelLocal(l),sandwichesDelLocal(\pre{l}))}
\asegura{mismosEmpleados(empleados(l) \masmas desempleados(l),empleados(\pre{l}) \masmas desempleados(\pre{l}))}
\asegura{(\forall \selector{b}{bebidaDistintos(combos(p))}) (stockBebidas(\pre{l}, b) \\ - cuentaBebida(b,combos(p))) == stockBebidas(l,b)}
\asegura{(\forall \selector{h}{sandwichDistintos(combos(p))}) (stockSandwiches(\pre{l}, h) \\ - cuentaSandwich(h,combos(p))) == stockSandwiches(l,b)}
\asegura{\IfThenElse{(energiaEmpleado(\pre{l},atendio(p)) - energiaPedido(p)) > 0}{ \\(energiaEmpleado(\pre{l},atendio(p)) - energiaPedido(p)) == energiaEmpleado(l,atendio(p))}{ \\mismosEmpleados(desempleados(l),agregarDesempleadoLocal(\pre{l},atendio(p))) \land \\ mismosEmpleados(empleados(l),despedirEmpleadoLocal(\pre{l},atendio(p))) }}
\asegura{\longitud{ventas(l)} == \longitud{ventas(\pre{l})} +1}
\asegura{numero(p) == mayorNumeroPedido(ventas(l))}
\asegura{mismosPedidos(ventas(l), ventas(\pre{l}) \masmas p )}

\end{problema}

\begin{problema}{candidatosAEmpleadosDelMesL}{l: Local}{[Empleado]}
\asegura{mismosEmpleados( \res ,mejoresEmpleados(l))}
\end{problema}


\begin{problema}{sancionL}{l: Local, e:Empleado, n:Energia}{}
\requiere[empleadoDelLocal]{empleadoPerteneceLocal(e,\pre{l})}
\modifica{l}
\asegura{mismos(bebidasDelLocal(l),bebidasdelLocal(\pre{l}))}
\asegura{mismos(sandwichesDelLocal(l),sandwichesDelLocal(\pre{l}))}
\asegura{mismosEmpleados(empleados(l) \masmas desempleados(l),empleados(\pre{l}) \masmas desempleados(\pre{l}))}
\asegura{mismos(ventas(l),ventas(\pre{l}))}
\asegura{(\forall \selector{b}{bebidasDelLocal(\pre{l})})stockBebidas(\pre{l}, b) == stockBebidas(l,b)}
\asegura{(\forall \selector{empleado}{empleados(\pre{l})}, e \neq empleado)energiaEmplado(\pre{l}, empleado) == \\ energiaEmpleado(l,empleado)}
\asegura{(\forall \selector{h}{sandwichesDelLocal(\pre{l})}) stockSandwiches(\pre{l}, h) == stockSandwiches(l,h)}
\asegura{\IfThenElse{(energiaEmpleado(\pre{l},e) - n) > 0}{ (energiaEmpleado(\pre{l},e) - n) == energiaEmpleado(l,e) \\ }{ mismosEmpleados(desempleados(l),agregarDesempleadoLocal(\pre{l},e)) \land \\ mismosEmpleados(empleados(l),despedirEmpleadoLocal(\pre{l},e)) }}
\end{problema}



\begin{problema}{elVagonetaL}{l: Local}{Empleado}
\end{problema}

\begin{problema}{anularPedidoL}{l: Local, n: \ent}{}
\requiere{esPedidoLocal(\pre{l},n)}
\requiere[empleadoDelLocal]{empleadoPerteneceLocal(atendio(pedidoLocal(\pre{l},n)),\pre{l})}
\modifica{l}
\asegura{mismos(bebidasDelLocal(l),bebidasdelLocal(\pre{l}))}
\asegura{mismos(sandwichesDelLocal(l),sandwichesDelLocal(\pre{l}))}
\asegura{mismosEmpleados(empleados(l),empleados(\pre{l}))}
\asegura{mismosEmpleados(desempleados(l),desempleados(\pre{l}))}
\asegura{\longitud{ventas(l)} == \longitud{ventas(\pre{l})} - 1}
\asegura{(\forall \selector{b}{bebidaDistintos(combos(pedidoLocal(\pre{l},n)))}) (stockBebidas(\pre{l}, b) \\ + cuentaBebida(b,combos(pedidoLocal(\pre{l},n)))) == stockBebidas(l,b)}
\asegura{(\forall \selector{h}{sandwichDistintos(combos(pedidoLocal(\pre{l},n)))}) (stockSandwiches(\pre{l}, h) \\ - cuentaSandwich(h,combos(pedidoLocal(\pre{l},n)))) == stockSandwiches(l,b)}
\asegura{(\forall \selector{e}{empleados(\pre{l})}, atendio(pedidoLocal(\pre{l},n)) \neq e)energiaEmplado(\pre{l}, e) == \\ energiaEmpleado(l,e)}
\asegura{ (energiaEmpleado(\pre{l},atendio(pedidoLocal(\pre{l},n))) + energiaPedido(pedidoLocal(\pre{l},n))) == energiaEmpleado(l,atendio(pedidoLocal(\pre{l},n)))
}
\asegura{mayorNumeroPedido(ventas(\pre{l})) - 1 == mayorNumeroPedido(ventas(l))}
\asegura{mismosPedidosNoNumero(ventas(l), eliminarPedidoVentasNumero(\pre{l}, n))}
\asegura{(\forall \selector{i}{\longitud{ventas(\pre{l})}}, numero(ventas(\pre{l})_{i}) \neq n ) \IfThenElse{ numero(ventas(\pre{l})_{i}) > n}{ \\ numero(ventas(\pre{l})_{i}) - 1 ==numero(ventas(l)_{i})}{numero(ventas(\pre{l})_{i}) ==numero(ventas(l)_{i}) } }

\end{problema}

\begin{problema}{agregarComboAlPedidoL}{l: Local, c: Combo, n:\ent}{}
\requiere{esPedidoLocal(\pre{l},n)}
\requiere[empleadoDelLocal]{empleadoPerteneceLocal(atendio(pedidoLocal(\pre{l},n)),\pre{l})}
\requiere[tieneEnergiaSuficiente]{energiaEmpleado(\pre{l},atendio(pedidoLocal(\pre{l},n)))) \geq \\ dificultad(c)}
\requiere[stockSandwichSuficiente]{(stockSandwiches(\pre{l}, sandwich(c)) - 1) > 0}
\requiere[stockBebidaSuficiente]{( stockBebidas(\pre{l}, bebida(c)) - 1) > 0}
\modifica{l}
\asegura{mismos(bebidasDelLocal(l),bebidasdelLocal(\pre{l}))}
\asegura{mismos(sandwichesDelLocal(l),sandwichesDelLocal(\pre{l}))}
\asegura{mismosEmpleados(empleados(l),empleados(\pre{l}))}
\asegura{mismosEmpleados(desempleados(l),desempleados(\pre{l}))}
\asegura{(\forall \selector{b}{bebidasDelLocal(\pre{l})}, b \neq bebida(c))stockBebidas(\pre{l}, b) == stockBebidas(l,b)}
\asegura{(\forall \selector{e}{empleados(\pre{l})}, e \neq atendio((pedidoLocal(\pre{l},n)))energiaEmplado(\pre{l}, e) == \\ energiaEmpleado(l,e)}
\asegura{(\forall \selector{h}{sandwichesDelLocal(\pre{l})}, h \neq sandwich(c)) stockSandwiches(\pre{l}, h) \\ == stockSandwiches(l,h)}
\asegura{stockBebidas(\pre{l}, bebida(c)) == stockBebidas(l,bebida(c)) + 1}
\asegura{stockSandwiches(\pre{l}, sandwich(c)) == stockSandwiches(l,sandwich(c)) + 1}
\asegura{energiaEmplado(\pre{l}, atendio(pedidoLocal(\pre{l},n))) == \\ energiaEmpleado(l,atendio(pedidoLocal(\pre{l},n))) + dificultad(c)}
\asegura{\longitud{ventas(l)} == \longitud{ventas(\pre{l})}}
\asegura{mismosPedidos( eliminarPedidoVentasNumero(\pre{l}, n), eliminarPedidoVentasNumero(l, n)) }
\asegura{ numero(pedidoLocal(\pre{l},n)) == numero(pedidoLocal(l,n))}
\asegura{ atendio(pedidoLocal(\pre{l},n)) == atendio(pedidoLocal(l,n))}
\asegura{ mismosCombosDePedidos(combos(pedidoLocal(\pre{l},n))++[c],combos(pedidoLocal(l,n)))}

 \end{problema}


\begin{problema}{stockBebidasL}{l: Local, b:Bebida}{Cantidad}
\requiere{ bebidaPerteneceLocal(b,l) } 
\asegura{ \res == stockBebidas(l,b)}
\end{problema}

\begin{problema}{stockSandwichesL}{l: Local, h:Hamburguesa}{Cantidad}
\requiere{ sandwichPerteneceLocal(h,l) } 
\asegura{ \res == stockSandwiches(l,h)}
\end{problema}

\begin{problema}{bebidasDelLocalL}{l: Local}{[Bebida]}
\asegura{ mismos (\res , bebidasDelLocal(l))}
\end{problema}
	
\begin{problema}{sandwichesDelLocalL}{l: Local}{[Hamburguesa]}
\asegura{ mismos (\res , sandwichesDelLocal(l))}
\end{problema}

\begin{problema}{empleadosL}{l: Local}{[Empleado]}
\asegura{ mismos (\res , empleados(l))}
\end{problema}

\begin{problema}{desempleadosL}{l: Local}{[Empleado]}
\asegura{ mismos (\res , desempleados(l))}
\end{problema}

\begin{problema}{energiaEmpleadoL}{l: Local, e:Empleado}{Energia}
\requiere{ empleadoPerteneceLocal(e,l) } 
\asegura{ \res == energiaEmpleado(l,e)}
\end{problema}

\begin{problema}{ventasL}{l: Local}{[Pedido]}
\asegura{ mismos (\res , ventas(l))}
\end{problema}

\begin{problema}{unaVentaCadaUno}{l:Local}{\bool}

\asegura{\res == lista}
%quienAtiendeVentasEmpleado(ventas(l),l) me da la lista de empleados actuales que atendieron los pedidos
\end{problema}

\begin{problema}{venderL}{l: Local, p:Pedido}{}
\requiere[numeroPedidoCorrecto]{numero(p) == mayorNumeroPedido(ventas(\pre{l})) + 1}
\requiere[empleadoDelLocal]{empleadoPerteneceLocal(atendio(p),\pre{l})}
\requiere[tieneEnergiaSuficiente]{energiaEmpleado(\pre{l},atendio(p)) \geq energiaPedido(p)} % No estoy seguro de este requiere
\requiere[stockSandwichSuficiente]{(\forall \selector{h}{sandwichDistintos(combos(p))}) ( stockSandwiches(\pre{l}, h) \\ - cuentaSandwich(h,combos(p))) > 0}
\requiere[stockBebidaSuficiente]{(\forall \selector{b}{bebidaDistintos(combos(p))}) ( stockBebidas(\pre{l}, b) \\ - cuentaBebida(b,combos(p))) > 0}
\modifica{l}
\asegura{mismos(bebidasDelLocal(l),bebidasdelLocal(\pre{l}))}
\asegura{mismos(sandwichesDelLocal(l),sandwichesDelLocal(\pre{l}))}
\asegura{mismosEmpleados(empleados(l) \masmas desempleados(l),empleados(\pre{l}) \masmas desempleados(\pre{l}))}
\asegura{(\forall \selector{b}{bebidaDistintos(combos(p))}) (stockBebidas(\pre{l}, b) \\ - cuentaBebida(b,combos(p))) == stockBebidas(l,b)}
\asegura{(\forall \selector{h}{sandwichDistintos(combos(p))}) (stockSandwiches(\pre{l}, h) \\ - cuentaSandwich(h,combos(p))) == stockSandwiches(l,b)}
\asegura{\IfThenElse{(energiaEmpleado(\pre{l},atendio(p)) - energiaPedido(p)) > 0}{ \\(energiaEmpleado(\pre{l},atendio(p)) - energiaPedido(p)) == energiaEmpleado(l,atendio(p))}{ \\mismosEmpleados(desempleados(l),agregarDesempleadoLocal(\pre{l},atendio(p))) \land \\ mismosEmpleados(empleados(l),despedirEmpleadoLocal(\pre{l},atendio(p))) }}
\asegura{\longitud{ventas(l)} == \longitud{ventas(\pre{l})} +1}
\asegura{numero(p) == mayorNumeroPedido(ventas(l))}
\asegura{mismosPedidos(ventas(l), ventas(\pre{l}) \masmas p )}

\end{problema}

\begin{problema}{candidatosAEmpleadosDelMesL}{l: Local}{[Empleado]}
\asegura{mismosEmpleados( \res ,mejoresEmpleados(l))}
\end{problema}


\begin{problema}{sancionL}{l: Local, e:Empleado, n:Energia}{}
\requiere[empleadoDelLocal]{empleadoPerteneceLocal(e,\pre{l})}
\modifica{l}
\asegura{mismos(bebidasDelLocal(l),bebidasdelLocal(\pre{l}))}
\asegura{mismos(sandwichesDelLocal(l),sandwichesDelLocal(\pre{l}))}
\asegura{mismosEmpleados(empleados(l) \masmas desempleados(l),empleados(\pre{l}) \masmas desempleados(\pre{l}))}
\asegura{mismos(ventas(l),ventas(\pre{l}))}
\asegura{(\forall \selector{b}{bebidasDelLocal(\pre{l})})stockBebidas(\pre{l}, b) == stockBebidas(l,b)}
\asegura{(\forall \selector{empleado}{empleados(\pre{l})}, e \neq empleado)energiaEmplado(\pre{l}, empleado) == \\ energiaEmpleado(l,empleado)}
\asegura{(\forall \selector{h}{sandwichesDelLocal(\pre{l})}) stockSandwiches(\pre{l}, h) == stockSandwiches(l,h)}
\asegura{\IfThenElse{(energiaEmpleado(\pre{l},e) - n) > 0}{ (energiaEmpleado(\pre{l},e) - n) == energiaEmpleado(l,e) \\ }{ mismosEmpleados(desempleados(l),agregarDesempleadoLocal(\pre{l},e)) \land \\ mismosEmpleados(empleados(l),despedirEmpleadoLocal(\pre{l},e)) }}
\end{problema}



\begin{problema}{elVagonetaL}{l: Local}{Empleado}
\end{problema}

\begin{problema}{anularPedidoL}{l: Local, n: \ent}{}
\requiere{esPedidoLocal(\pre{l},n)}
\requiere[empleadoDelLocal]{empleadoPerteneceLocal(atendio(pedidoLocal(\pre{l},n)),\pre{l})}
\modifica{l}
\asegura{mismos(bebidasDelLocal(l),bebidasdelLocal(\pre{l}))}
\asegura{mismos(sandwichesDelLocal(l),sandwichesDelLocal(\pre{l}))}
\asegura{mismosEmpleados(empleados(l),empleados(\pre{l}))}
\asegura{mismosEmpleados(desempleados(l),desempleados(\pre{l}))}
\asegura{\longitud{ventas(l)} == \longitud{ventas(\pre{l})} - 1}
\asegura{(\forall \selector{b}{bebidaDistintos(combos(pedidoLocal(\pre{l},n)))}) (stockBebidas(\pre{l}, b) \\ + cuentaBebida(b,combos(pedidoLocal(\pre{l},n)))) == stockBebidas(l,b)}
\asegura{(\forall \selector{h}{sandwichDistintos(combos(pedidoLocal(\pre{l},n)))}) (stockSandwiches(\pre{l}, h) \\ - cuentaSandwich(h,combos(pedidoLocal(\pre{l},n)))) == stockSandwiches(l,b)}
\asegura{(\forall \selector{e}{empleados(\pre{l})}, atendio(pedidoLocal(\pre{l},n)) \neq e)energiaEmplado(\pre{l}, e) == \\ energiaEmpleado(l,e)}
\asegura{ (energiaEmpleado(\pre{l},atendio(pedidoLocal(\pre{l},n))) + energiaPedido(pedidoLocal(\pre{l},n))) == energiaEmpleado(l,atendio(pedidoLocal(\pre{l},n)))
}
\asegura{mayorNumeroPedido(ventas(\pre{l})) - 1 == mayorNumeroPedido(ventas(l))}
\asegura{mismosPedidosNoNumero(ventas(l), eliminarPedidoVentasNumero(\pre{l}, n))}
\asegura{(\forall \selector{i}{\longitud{ventas(\pre{l})}}, numero(ventas(\pre{l})_{i}) \neq n ) \IfThenElse{ numero(ventas(\pre{l})_{i}) > n}{ \\ numero(ventas(\pre{l})_{i}) - 1 ==numero(ventas(l)_{i})}{numero(ventas(\pre{l})_{i}) ==numero(ventas(l)_{i}) } }

\end{problema}

\begin{problema}{agregarComboAlPedidoL}{l: Local, c: Combo, n:\ent}{}
\requiere{esPedidoLocal(\pre{l},n)}
\requiere[empleadoDelLocal]{empleadoPerteneceLocal(atendio(pedidoLocal(\pre{l},n)),\pre{l})}
\requiere[tieneEnergiaSuficiente]{energiaEmpleado(\pre{l},atendio(pedidoLocal(\pre{l},n)))) \geq \\ dificultad(c)}
\requiere[stockSandwichSuficiente]{(stockSandwiches(\pre{l}, sandwich(c)) - 1) > 0}
\requiere[stockBebidaSuficiente]{( stockBebidas(\pre{l}, bebida(c)) - 1) > 0}
\modifica{l}
\asegura{mismos(bebidasDelLocal(l),bebidasdelLocal(\pre{l}))}
\asegura{mismos(sandwichesDelLocal(l),sandwichesDelLocal(\pre{l}))}
\asegura{mismosEmpleados(empleados(l),empleados(\pre{l}))}
\asegura{mismosEmpleados(desempleados(l),desempleados(\pre{l}))}
\asegura{(\forall \selector{b}{bebidasDelLocal(\pre{l})}, b \neq bebida(c))stockBebidas(\pre{l}, b) == stockBebidas(l,b)}
\asegura{(\forall \selector{e}{empleados(\pre{l})}, e \neq atendio((pedidoLocal(\pre{l},n)))energiaEmplado(\pre{l}, e) == \\ energiaEmpleado(l,e)}
\asegura{(\forall \selector{h}{sandwichesDelLocal(\pre{l})}, h \neq sandwich(c)) stockSandwiches(\pre{l}, h) \\ == stockSandwiches(l,h)}
\asegura{stockBebidas(\pre{l}, bebida(c)) == stockBebidas(l,bebida(c)) + 1}
\asegura{stockSandwiches(\pre{l}, sandwich(c)) == stockSandwiches(l,sandwich(c)) + 1}
\asegura{energiaEmplado(\pre{l}, atendio(pedidoLocal(\pre{l},n))) == \\ energiaEmpleado(l,atendio(pedidoLocal(\pre{l},n))) + dificultad(c)}
\asegura{\longitud{ventas(l)} == \longitud{ventas(\pre{l})}}
\asegura{mismosPedidos( eliminarPedidoVentasNumero(\pre{l}, n), eliminarPedidoVentasNumero(l, n)) }
\asegura{ numero(pedidoLocal(\pre{l},n)) == numero(pedidoLocal(l,n))}
\asegura{ atendio(pedidoLocal(\pre{l},n)) == atendio(pedidoLocal(l,n))}
\asegura{ mismosCombosDePedidos(combos(pedidoLocal(\pre{l},n))++[c],combos(pedidoLocal(l,n)))}

 \end{problema}

\newpage

\section{Funciones Auxiliares} 

\aux{distintos}{ls:[T]}{\bool}{ 
  (\forall i,j \selec [0..|ls|), i \neq j) ls_i \neq ls_j
}


\aux{energiaEnRango}{e: Energia} {\bool}{
        0 \leq e \leq 100
}

\subsection{Combo}
% los aux del tipo combo

\subsection{Pedido}

% los aux del tipo pedido

\subsection{Local}

% los aux del tipo local






\end{document} %Termin�!


%Muestra de como poner las cosas para el TPI

%\section{Implementaci\'on}




%\subsection{Flor}

%\lstinputlisting[caption=Flor.h]{../src/Flor.h}

%\lstinputlisting[caption=Flor.cpp]{../src/Flor.cpp}

%\newpage



