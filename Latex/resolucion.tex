\documentclass[10pt,a4paper,spanish]{article} 

\usepackage{a4wide}
\usepackage{amsmath, amscd, amsthm, latexsym}
\usepackage[spanish]{babel} % Le indicamos a LaTeX que vamos a escribir en espa�ol.
\usepackage[latin1]{inputenc} % Permite utilizar tildes y e�es normalmente
\usepackage{soul}
%\usepackage[utf8]{inputenc}
%\usepackage{bbm}
\usepackage{framed}
\usepackage{listings}
\usepackage{enumerate}
\usepackage{xspace}
\usepackage{longtable}
\usepackage{caratula}
\usepackage{fancyhdr}
\pagestyle{fancyplain}
%\overfullrule=2cm %Ver margenes en negro para overfull \hbox
\usepackage{ifthen}
\usepackage{amssymb}
\usepackage{multicol}
\usepackage{graphicx}
\usepackage[absolute]{textpos}
\makeatletter

\@ifclassloaded{beamer}{%
  \newcommand{\tocarEspacios}{%
    \addtolength{\leftskip}{4em}%
    \addtolength{\parindent}{-3em}%
  }%
}
{%
  \usepackage[top=1cm,bottom=2cm,left=1cm,right=1cm]{geometry}%
  \usepackage{color}%
  \newcommand{\tocarEspacios}{%
    \addtolength{\leftskip}{5em}%
    \addtolength{\parindent}{-3em}%
  }%
}

\newcommand{\encabezadoDeProblema}[4]{%
  % Ponemos la palabrita problema en tt
%  \noindent%
  {\normalfont\bfseries\ttfamily problema}%
  % Ponemos el nombre del problema
  \ %
  {\normalfont\ttfamily #2}%
  \ 
  % Ponemos los parametros
  (#3)%
  \ifthenelse{\equal{#4}{}}{}{%
  \ =\ %
  % Ponemos el nombre del resultado
  {\normalfont\ttfamily #1}%
  % Por ultimo, va el tipo del resultado
  \ : #4}
}

\newcommand{\encabezadoDeTipo}[2]{%
  % Ponemos la palabrita tipo en tt
  {\normalfont\bfseries\ttfamily tipo}%
  % Ponemos el nombre del tipo
  \ %
  {\normalfont\ttfamily #2}%
  \ifthenelse{\equal{#1}{}}{}{$\langle$#1$\rangle$}
}

% Primero definiciones de cosas al estilo title, author, date

\def\materia#1{\gdef\@materia{#1}}
\def\@materia{No especifi\'o la materia}
\def\lamateria{\@materia}

\def\cuatrimestre#1{\gdef\@cuatrimestre{#1}}
\def\@cuatrimestre{No especifi\'o el cuatrimestre}
\def\elcuatrimestre{\@cuatrimestre}

\def\anio#1{\gdef\@anio{#1}}
\def\@anio{No especifi\'o el anio}
\def\elanio{\@anio}

\def\fecha#1{\gdef\@fecha{#1}}
\def\@fecha{\today}
\def\lafecha{\@fecha}

\def\nombre#1{\gdef\@nombre{#1}}
\def\@nombre{No especific'o el nombre}
\def\elnombre{\@nombre}

\def\practicas#1{\gdef\@practica{#1}}
\def\@practica{No especifi\'o el n\'umero de pr\'actica}
\def\lapractica{\@practica}


% Esta macro convierte el numero de cuatrimestre a palabras
\newcommand{\cuatrimestreLindo}{
  \ifthenelse{\equal{\elcuatrimestre}{1}}
  {Primer cuatrimestre}
  {\ifthenelse{\equal{\elcuatrimestre}{2}}
  {Segundo cuatrimestre}
  {Verano}}
}


\newcommand{\depto}{{UBA -- Facultad de Ciencias Exactas y Naturales --
      Departamento de Computaci\'on}}

\newcommand{\titulopractica}{
  \centerline{\depto}
  \vspace{1ex}
  \centerline{{\Large\lamateria}}
  \vspace{0.5ex}
  \centerline{\cuatrimestreLindo de \elanio}
  \vspace{2ex}
  \centerline{{\huge Pr\'actica \lapractica -- \elnombre}}
  \vspace{5ex}
  \arreglarincisos
  \newcounter{ejercicio}
  \newenvironment{ejercicio}{\stepcounter{ejercicio}\textbf{Ejercicio
      \theejercicio}%
    \renewcommand\@currentlabel{\theejercicio}%
  }{\vspace{0.2cm}}
}  


\newcommand{\titulotp}{
  \centerline{\depto}
  \vspace{1ex}
  \centerline{{\Large\lamateria}}
  \vspace{0.5ex}
  \centerline{\cuatrimestreLindo de \elanio}
  \vspace{0.5ex}
  \centerline{\lafecha}
  \vspace{2ex}
  \centerline{{\huge\elnombre}}
  \vspace{5ex}
}


%practicas
\newcommand{\practica}[2]{%
    \title{Pr\'actica #1 \\ #2}
    \author{Algoritmos y Estructuras de Datos I}
    \date{Segundo Cuatrimestre 2015}

    \maketitlepractica{#1}{#2}
}

\newcommand \maketitlepractica[2] {%
\begin{center}
\begin{tabular}{r cr}
 \begin{tabular}{c}
{\large\bf\textsf{\ Algoritmos y Estructuras de Datos I\ }}\\ 
Segundo Cuatrimestre 2015\\
\title{\normalsize Gu\'ia Pr\'actica #1 \\ \textbf{#2}}\\
\@title
\end{tabular} &
\begin{tabular}{@{} p{1.6cm} @{}}
\includegraphics[width=1.6cm]{logodpt.jpg}
\end{tabular} &
\begin{tabular}{l @{}}
 \emph{Departamento de Computaci\'on} \\
 \emph{Facultad de Ciencias Exactas y Naturales} \\
 \emph{Universidad de Buenos Aires} \\
\end{tabular} 
\end{tabular}
\end{center}

\bigskip
}


% Simbolos varios

\newcommand{\ent}{\ensuremath{\mathbb{Z}}}
\newcommand{\float}{\ensuremath{\mathbb{R}}}
\newcommand{\bool}{\ensuremath{\mathsf{Bool}}}
\newcommand{\True}{\ensuremath{\mathrm{True}}}
\newcommand{\False}{\ensuremath{\mathrm{False}}}
\newcommand{\Then}{\ensuremath{\rightarrow}}
\newcommand{\Iff}{\ensuremath{\leftrightarrow}}
\newcommand{\implica}{\ensuremath{\longrightarrow}}
\newcommand{\IfThenElse}[3]{\ensuremath{\mathsf{if}\ #1\ \mathsf{then}\ #2\ \mathsf{else}\ #3}}


\newcommand{\rango}[2]{[#1\twodots#2]}
\newcommand{\comp}[2]{[\,#1\,|\,#2\,]}

\newcommand{\rangoac}[2]{(#1\twodots#2]}
\newcommand{\rangoca}[2]{[#1\twodots#2)}
\newcommand{\rangoaa}[2]{(#1\twodots#2)}

%ejercicios
\newtheorem{exercise}{Ejercicio}
\newenvironment{ejercicio}{\begin{exercise}\rm}{\end{exercise} \vspace{0.2cm}}
\newenvironment{items}{\begin{enumerate}[i)]}{\end{enumerate}}
\newenvironment{subitems}{\begin{enumerate}[a)]}{\end{enumerate}}
\newcommand{\sugerencia}[1]{\noindent \textbf{Sugerencia:} #1}

%tipos basicos
\newcommand{\rea}{\ensuremath{\mathsf{Float}}}
\newcommand{\cha}{\ensuremath{\mathsf{Char}}}

\newcommand{\mcd}{\mathrm{mcd}}
\newcommand{\prm}[1]{\ensuremath{\mathsf{prm}(#1)}}
\newcommand{\sgd}[1]{\ensuremath{\mathsf{sgd}(#1)}}

%listas
\newcommand{\TLista}[1]{[#1]}
\newcommand{\lvacia}{\ensuremath{[\ ]}}
\newcommand{\lv}{\ensuremath{[\ ]}}
\newcommand{\longitud}[1]{\left| #1 \right|}
\newcommand{\cons}[1]{\ensuremath{\mathsf{cons}}(#1)}
\newcommand{\indice}[1]{\ensuremath{\mathsf{indice}}(#1)}
\newcommand{\conc}[1]{\ensuremath{\mathsf{conc}}(#1)}
\newcommand{\cab}[1]{\ensuremath{\mathsf{cab}}(#1)}
\newcommand{\cola}[1]{\ensuremath{\mathsf{cola}}(#1)}
\newcommand{\sub}[1]{\ensuremath{\mathsf{sub}}(#1)}
\newcommand{\en}[1]{\ensuremath{\mathsf{en}}(#1)}
\newcommand{\cuenta}[2]{\mathsf{cuenta}\ensuremath{(#1, #2)}}
\newcommand{\suma}[1]{\mathsf{suma}(#1)}
\newcommand{\twodots}{\ensuremath{\mathrm{..}}}
\newcommand{\masmas}{\ensuremath{++}}

% Acumulador
\newcommand{\acum}[1]{\ensuremath{\mathsf{acum}}(#1)}
\newcommand{\acumselec}[3]{\ensuremath{\mathrm{acum}(#1 |  #2, #3)}}

% \selector{variable}{dominio}
\newcommand{\selector}[2]{#1~\ensuremath{\leftarrow}~#2}
\newcommand{\selec}{\ensuremath{\leftarrow}}


\newenvironment{problema}[4][res]{%
  % El parametro 1 (opcional) es el nombre del resultado
  % El parametro 2 es el nombre del problema
  % El parametro 3 son los parametros
  % El parametro 4 es el tipo del resultado
  % Preambulo del ambiente problema
  % Tenemos que definir los comandos requiere, asegura, modifica y aux
  \newcommand{\requiere}[2][]{%
    {\normalfont\bfseries\ttfamily requiere}%
    \ifthenelse{\equal{##1}{}}{}{\ {\normalfont\ttfamily ##1} :}\ %
    \ensuremath{##2}%
    {\normalfont\bfseries\,;\par}%
  }
  \newcommand{\asegura}[2][]{%
    {\normalfont\bfseries\ttfamily asegura}%
    \ifthenelse{\equal{##1}{}}{}{\ {\normalfont\ttfamily ##1} :}\
    \ensuremath{##2}%
    {\normalfont\bfseries\,;\par}%
  }
  \newcommand{\modifica}[1]{%
    {\normalfont\bfseries\ttfamily modifica\ }%
    \ensuremath{##1}%
    {\normalfont\bfseries\,;\par}%
  }
  \renewcommand{\aux}[4]{%
    {\normalfont\bfseries\ttfamily aux\ }%
    {\normalfont\ttfamily ##1}%
    \ifthenelse{\equal{##2}{}}{}{\ (##2)}\ : ##3\, = \ensuremath{##4}%
    {\normalfont\bfseries\,;\par}%
  }
  \newcommand{\res}{#1}
  \vspace{1ex}
  \noindent
  \encabezadoDeProblema{#1}{#2}{#3}{#4}
  % Abrimos la llave
  \{\par%
  \tocarEspacios
}
% Ahora viene el cierre del ambiente problema
{
  % Cerramos la llave
  \noindent\}
  \vspace{1ex}
}


  \newcommand{\aux}[4]{%
    {\normalfont\bfseries\ttfamily aux\ }%
    {\normalfont\ttfamily #1}%
    \ifthenelse{\equal{#2}{}}{}{\ (#2)}\ : #3\, = \ensuremath{#4}%
    {\normalfont\bfseries\,;\par}%
  }


\newcommand{\pre}[1]{\textsf{pre}\ensuremath{(#1)}}

\newcommand{\problemanom}[1]{\textsf{#1}}
\newcommand{\problemail}[3]{\textsf{problema #1}\ensuremath{(#2) = #3}}
\newcommand{\problemailsinres}[2]{\textsf{problema #1}\ensuremath{(#2)}}
\newcommand{\requiereil}[2]{\textsf{requiere #1: }\ensuremath{#2}}
\newcommand{\asegurail}[2]{\textsf{asegura #1: }\ensuremath{#2}}
\newcommand{\modificail}[1]{\textsf{modifica }\ensuremath{#1}}
\newcommand{\auxil}[2]{\textsf{aux }\ensuremath{#1 = #2}}
\newcommand{\auxilc}[4]{\textsf{aux }\ensuremath{#1( #2 ): #3 = #4}}
\newcommand{\auxnom}[1]{\textsf{aux }\ensuremath{#1}}

\newcommand{\comentario}[1]{{/*\ #1\ */}}

\newcommand{\nom}[1]{\ensuremath{\mathsf{#1}}}

% -----------------
% Tipos compuestos
% -----------------

\newcommand{\Pred}[1]{\mathit{#1}}
\newcommand{\TSet}[1]{\textsf{Conjunto}\ensuremath{\langle #1 \rangle}}
\newcommand{\TSetFinito}[1]{\textsf{Conjunto}\ensuremath{\langle #1 \rangle}}
\newcommand{\TRac}{\tiponom{Racional}}
\newcommand{\TVec}{\tiponom{Vector}}
\newcommand{\Func}[1]{\mathrm{#1}}
\newcommand{\cardinal}[1]{\left| #1 \right|}


\newcommand{\sinonimo}[2]{%
  \noindent%
  {\normalfont\bfseries\ttfamily tipo\ }%
  #1\ =\ #2%
  {\normalfont\bfseries\,;\par}
}

\newcommand{\enum}[2]{%
  \noindent%
  {\normalfont\bfseries\ttfamily tipo\ }%
  #1\ =\ #2%
  {\normalfont\bfseries\,;\par}
}

%~ \newenvironment{tipo}[1]{%
    %~ \vspace{0.2cm}
    %~ \textsf{tipo #1}\ensuremath{\{}\\
    %~ \begin{tabular}[l]{p{0.02\textwidth} p{0.02\textwidth} p{0.82 \textwidth}}
%~ }{%
    %~ \end{tabular}
%~ 
    %~ \ensuremath{\}}
    %~ \vspace{0.15cm}
%~ }
%~ 

\newenvironment{tipo}[2][]{%
  % Preambulo del ambiente tipo
  % Tenemos que definir los comandos observador (con requiere) y aux
  \newcommand{\observador}[3]{%
    {\normalfont\bfseries\ttfamily observador\ }%
    {\normalfont\ttfamily ##1}%
    \ifthenelse{\equal{##2}{}}{}{\ (##2)}\ : ##3%
    {\normalfont\bfseries\,;\par}%
  }
  \newcommand{\requiere}[2][]{{%
    \addtolength{\leftskip}{3em}%
    \setlength{\parindent}{-2em}%
    {\normalfont\bfseries\ttfamily requiere}%
    \ifthenelse{\equal{##1}{}}{}{\ {\normalfont\ttfamily ##1} :}\ 
    \ensuremath{##2}%
    {\normalfont\bfseries\,;\par}}
  }
  \newcommand{\explicacion}[1]{{%
    \addtolength{\leftskip}{3em}%
    \setlength{\parindent}{-2em}%
    \par \hspace{2.3em} ##1 %
    {\par}
    }
  }
  \newcommand{\invariante}[2][]{%
    {\normalfont\bfseries\ttfamily invariante}%
    \ifthenelse{\equal{##1}{}}{}{\ {\normalfont\ttfamily ##1} :}\ 
    \ensuremath{##2}%
    {\normalfont\bfseries\,;\par}%
  }
  \renewcommand{\aux}[4]{%
    {\normalfont\bfseries\ttfamily aux\ }%
    {\normalfont\ttfamily ##1}%
    \ifthenelse{\equal{##2}{}}{}{\ (##2)}\ : ##3\, = \ensuremath{##4}%
    {\normalfont\bfseries\,;\par}%
  }
  \vspace{1ex}
  \noindent
  \encabezadoDeTipo{#1}{#2}
  % Abrimos la llave
  \{\par%
  \tocarEspacios
}
% Ahora viene el cierre del ambiente tipo
{
  % Cerramos la llave
  \noindent\}
  \vspace{1ex}
}


%~ \newcommand{\observador}[3]{%
    %~ & \multicolumn{2}{p{0.85\textwidth}}{\textsf{observador #1}\ensuremath{(#2):#3}}\\%
    %~ }
    
%~ \newcommand{\observador}[3]{%
    %~ {\normalfont\bfseries\ttfamily observador\ }%
    %~ {\normalfont\ttfamily ##1}%
    %~ \ifthenelse{\equal{##2}{}}{}{\ (##2)}\ : ##3%
    %~ {\normalfont\bfseries\,;\par}%
%~ }
    

%~ \newcommand{\observadorconreq}[3]{
    %~ & \multicolumn{2}{p{0.85\textwidth}}{\textsf{observador #1}\ensuremath{(#2):#3 \{}}\\
%~ }
%~ \newcommand{\observadorconreqfin}{
    %~ & \multicolumn{2}{p{0.85\textwidth}}{\ensuremath{\}}}\\
%~ }
%~ \newcommand{\obsrequiere}[2][]{& & \textsf{requiere #1: }\ensuremath{#2};\\}
%~ 
%~ \newcommand{\explicacion}[1]{&& #1 \\}
%~ \newcommand{\invariante}[2][]{%
    %~ & \multicolumn{2}{p{0.85\textwidth}}{\textsf{invariante #1: }\ensuremath{#2}}\\%
%~ }
%~ \newcommand{\auxinvariante}[2]{
    %~ & \multicolumn{2}{p{0.85\textwidth}}{\textsf{aux }\ensuremath{#1 = #2}};\\
%~ }
%~ \newcommand{\auxiliar}[4]{
    %~ & \multicolumn{2}{p{0.85\textwidth}}{\textsf{aux }\ensuremath{#1(#2): #3 = #4}};\\
%~ }

\newcommand{\tiponom}[1]{\ensuremath{\mathsf{#1}}\xspace}
\newcommand{\obsnom}[1]{\ensuremath{\mathsf{#1}}}

% -----------------
% Ecuaciones de terminacion en funcional
% -----------------

\newenvironment{ecuaciones}{%
    $$
    \begin{array}{l @{\ /\ (} l @{,\ } l @{)\ =\ } l}
}{%
    \end{array}
    $$
}




\newcommand{\ecuacion}[4]{#1 & #2 & #3 & #4\\}

\newcommand{\concat}{\nom{concat}}

% Listas por comprension. El primer parametro es la expresion y el
% segundo tiene los selectores y las condiciones.
%*\newcommand{\comp}[2]{[\,#1\,|\,#2\,]}























% En las practicas/parciales usamos numeros arabigos para los ejercicios.
% Aca cambiamos los enumerate comunes para que usen letras y numeros
% romanos
\newcommand{\arreglarincisos}{%
  \renewcommand{\theenumi}{\alph{enumi}}
  \renewcommand{\theenumii}{\roman{enumii}}
  \renewcommand{\labelenumi}{\theenumi)}
  \renewcommand{\labelenumii}{\theenumii)}
}





%%%%%%%%%%%%%%%%%%%%%%%%%%%%%% PARCIAL %%%%%%%%%%%%%%%%%%%%%%%%
\let\@xa\expandafter
\newcommand{\tituloparcial}{\centerline{\depto -- \lamateria}
  \centerline{\elnombre -- \lafecha}%
  \setlength{\TPHorizModule}{10mm} % Fija las unidades de textpos
  \setlength{\TPVertModule}{\TPHorizModule} % Fija las unidades de
                                % textpos
  \arreglarincisos
  \newcounter{total}% Este contador va a guardar cuantos incisos hay
                    % en el parcial. Si un ejercicio no tiene incisos,
                    % cuenta como un inciso.
  \newcounter{contgrilla} % Para hacer ciclos
  \newcounter{columnainicial} % Se van a usar para los cline cuando un
  \newcounter{columnafinal}   % ejercicio tenga incisos.
  \newcommand{\primerafila}{}
  \newcommand{\segundafila}{}
  \newcommand{\rayitas}{} % Esto va a guardar los \cline de los
                          % ejercicios con incisos, asi queda mas bonito
  \newcommand{\anchodegrilla}{20} % Es para textpos
  \newcommand{\izquierda}{7} % Estos dos le dicen a textpos donde colocar
  \newcommand{\abajo}{2}     % la grilla
  \newcommand{\anchodecasilla}{0.4cm}
  \setcounter{columnainicial}{1}
  \setcounter{total}{0}
  \newcounter{ejercicio}
  \setcounter{ejercicio}{0}
  \renewenvironment{ejercicio}[1]
  {%
    \stepcounter{ejercicio}\textbf{\noindent Ejercicio \theejercicio. [##1
      puntos]}% Formato
    \renewcommand\@currentlabel{\theejercicio}% Esto es para las
                                % referencias
    \newcommand{\invariante}[2]{%
      {\normalfont\bfseries\ttfamily invariante}%
      \ ####1\hspace{1em}####2%
    }%
    \renewcommand{\problema}[5][result]{
      \encabezadoDeProblema{####1}{####2}{####3}{####4}\hspace{1em}####5}%
  }% Aca se termina el principio del ejercicio
  {% Ahora viene el final
    % Esto suma la cantidad de incisos o 1 si no hubo ninguno
    \ifthenelse{\equal{\value{enumi}}{0}}
    {\addtocounter{total}{1}}
    {\addtocounter{total}{\value{enumi}}}
    \ifthenelse{\equal{\value{ejercicio}}{1}}{}
    {
      \g@addto@macro\primerafila{&} % Si no estoy en el primer ej.
      \g@addto@macro\segundafila{&}
    }
    \ifthenelse{\equal{\value{enumi}}{0}}
    {% No tiene incisos
      \g@addto@macro\primerafila{\multicolumn{1}{|c|}}
      \bgroup% avoid overwriting somebody else's value of \tmp@a
      \protected@edef\tmp@a{\theejercicio}% expand as far as we can
      \@xa\g@addto@macro\@xa\primerafila\@xa{\tmp@a}%
      \egroup% restore old value of \tmp@a, effect of \g@addto.. is
      
      \stepcounter{columnainicial}
    }
    {% Tiene incisos
      % Primero ponemos el encabezado
      \g@addto@macro\primerafila{\multicolumn}% Ahora el numero de items
      \bgroup% avoid overwriting somebody else's value of \tmp@a
      \protected@edef\tmp@a{\arabic{enumi}}% expand as far as we can
      \@xa\g@addto@macro\@xa\primerafila\@xa{\tmp@a}%
      \egroup% restore old value of \tmp@a, effect of \g@addto.. is
      % global 
      % Ahora el formato
      \g@addto@macro\primerafila{{|c|}}%
      % Ahora el numero de ejercicio
      \bgroup% avoid overwriting somebody else's value of \tmp@a
      \protected@edef\tmp@a{\theejercicio}% expand as far as we can
      \@xa\g@addto@macro\@xa\primerafila\@xa{\tmp@a}%
      \egroup% restore old value of \tmp@a, effect of \g@addto.. is
      % global 
      % Ahora armamos la segunda fila
      \g@addto@macro\segundafila{\multicolumn{1}{|c|}{a}}%
      \setcounter{contgrilla}{1}
      \whiledo{\value{contgrilla}<\value{enumi}}
      {%
        \stepcounter{contgrilla}
        \g@addto@macro\segundafila{&\multicolumn{1}{|c|}}
        \bgroup% avoid overwriting somebody else's value of \tmp@a
        \protected@edef\tmp@a{\alph{contgrilla}}% expand as far as we can
        \@xa\g@addto@macro\@xa\segundafila\@xa{\tmp@a}%
        \egroup% restore old value of \tmp@a, effect of \g@addto.. is
        % global 
      }
      % Ahora armo las rayitas
      \setcounter{columnafinal}{\value{columnainicial}}
      \addtocounter{columnafinal}{-1}
      \addtocounter{columnafinal}{\value{enumi}}
      \bgroup% avoid overwriting somebody else's value of \tmp@a
      \protected@edef\tmp@a{\noexpand\cline{%
          \thecolumnainicial-\thecolumnafinal}}%
      \@xa\g@addto@macro\@xa\rayitas\@xa{\tmp@a}%
      \egroup% restore old value of \tmp@a, effect of \g@addto.. is
      \setcounter{columnainicial}{\value{columnafinal}}
      \stepcounter{columnainicial}
    }
    \setcounter{enumi}{0}%
    \vspace{0.2cm}%
  }%
  \newcommand{\tercerafila}{}
  \newcommand{\armartercerafila}{
    \setcounter{contgrilla}{1}
    \whiledo{\value{contgrilla}<\value{total}}
    {\stepcounter{contgrilla}\g@addto@macro\tercerafila{&}}
  }
  \newcommand{\grilla}{%
    \g@addto@macro\primerafila{&\textbf{TOTAL}}
    \g@addto@macro\segundafila{&}
    \g@addto@macro\tercerafila{&}
    \armartercerafila
    \ifthenelse{\equal{\value{total}}{\value{ejercicio}}}
    {% No hubo incisos
      \begin{textblock}{\anchodegrilla}(\izquierda,\abajo)
        \begin{tabular}{|*{\value{total}}{p{\anchodecasilla}|}c|}
          \hline
          \primerafila\\
          \hline
          \tercerafila\\
          \tercerafila\\
          \hline
        \end{tabular}
      \end{textblock}
    }
    {% Hubo incisos
      \begin{textblock}{\anchodegrilla}(\izquierda,\abajo)
        \begin{tabular}{|*{\value{total}}{p{\anchodecasilla}|}c|}
          \hline
          \primerafila\\
          \rayitas
          \segundafila\\
          \hline
          \tercerafila\\
          \tercerafila\\
          \hline
        \end{tabular}
      \end{textblock}
    }
  }%
  \vspace{0.4cm}
  \textbf{Nro. de orden:}
  
  \textbf{LU:}
  
  \textbf{Apellidos:}
  
  \textbf{Nombres:}
  \vspace{0.5cm}
}



% AMBIENTE CONSIGNAS
% Se usa en el TP para ir agregando las cosas que tienen que resolver
% los alumnos.
% Dentro del ambiente hay que usar \item para cada consigna

\newcounter{consigna}
\setcounter{consigna}{0}

\newenvironment{consignas}{%
  \newcommand{\consigna}{\stepcounter{consigna}\textbf{\theconsigna.}}%
  \renewcommand{\ejercicio}[1]{\item ##1 }
  \renewcommand{\problema}[5][result]{\item
    \encabezadoDeProblema{##1}{##2}{##3}{##4}\hspace{1em}##5}%
  \newcommand{\invariante}[2]{\item%
    {\normalfont\bfseries\ttfamily invariante}%
    \ ##1\hspace{1em}##2%
  }
  \renewcommand{\aux}[4]{\item%
    {\normalfont\bfseries\ttfamily aux\ }%
    {\normalfont\ttfamily ##1}%
    \ifthenelse{\equal{##2}{}}{}{\ (##2)}\ : ##3 \hspace{1em}##4%
  }
  % Comienza la lista de consignas
  \begin{list}{\consigna}{%
      \setlength{\itemsep}{0.5em}%
      \setlength{\parsep}{0cm}%
    }
}%
{\end{list}}



% para decidir si usar && o ^
\newcommand{\y}[0]{\ensuremath{\land}}

% macros de correctitud
\newcommand{\semanticComment}[2]{#1 \ensuremath{#2};}
\newcommand{\namedSemanticComment}[3]{#1 #2: \ensuremath{#3};}


\newcommand{\local}[1]{\semanticComment{local}{#1}}

\newcommand{\vale}[1]{\semanticComment{vale}{#1}}
\newcommand{\valeN}[2]{\namedSemanticComment{vale}{#1}{#2}}
\newcommand{\impl}[1]{\semanticComment{implica}{#1}}
\newcommand{\implN}[2]{\namedSemanticComment{implica}{#1}{#2}}
\newcommand{\estado}[1]{\semanticComment{estado}{#1}}

\newcommand{\invarianteCN}[2]{\namedSemanticComment{invariante}{#1}{#2}}
\newcommand{\invarianteC}[1]{\semanticComment{invariante}{#1}}
\newcommand{\varianteCN}[2]{\namedSemanticComment{variante}{#1}{#2}}
\newcommand{\varianteC}[1]{\semanticComment{variante}{#1}}
% Macros especificas para especificar problemas en AyEDI
%\usepackage{listings}
\usepackage{xcolor}

\definecolor{gray97}{gray}{.97}
\definecolor{gray75}{gray}{.75}
\definecolor{gray45}{gray}{.45}

\lstset{
	language=c++,
	frame=Ltb,
	framerule=0pt,
	aboveskip=0.5cm,
	framextopmargin=3pt,
	framexbottommargin=3pt,
	framexleftmargin=0.4cm,
	framesep=0pt,
	rulesep=.4pt,
	backgroundcolor=\color{gray97},
	rulesepcolor=\color{black},
	%
	stringstyle=\sffamily,
	showstringspaces = false,
	basicstyle=\small\sffamily,
	commentstyle=\color{gray45} \itshape,
	keywordstyle=\bfseries,
	%
	numbers=left,
	numbersep=15pt,
	numberstyle=\tiny,
	numberfirstline = false,
	breaklines=true
}
\renewcommand{\lstlistingname}{Archivo}

\lstnewenvironment{listing}[1][]
   {\lstset{#1}\pagebreak[0]}{\pagebreak[0]} Para el TPI

\newcommand{\comen}[2]{%
\begin{framed}
\noindent \textsf{#1:} #2
\end{framed}
}
% Aca solo vamos a poner el esqueleto del documento, pero no vamos a especificar nada.

% Encabezado
\lhead{Algoritmos y Estructuras de Datos I}
\rhead{Grupo 19 - Plaza Sesamo}
% Pie de pagina
\renewcommand{\footrulewidth}{0.4pt}
\lfoot{Facultad de Ciencias Exactas y Naturales}
\rfoot{Universidad de Buenos Aires}


\begin{document} % Todo lo que escribamos a partir de aca va a aparecer en el documento.

% Datos de caratula
\materia{Algoritmos y Estructura de Datos I}
\titulo{Trabajo Pr\'actico 1 - Especificaci\'on}
\subtitulo{Fat Food}
\grupo{Grupo 19 Plaza Sesamo}
\fechas{\today}
\integrante{Gagliardi, Hernan Gabriel}{810/12}{hg.gagliardi@gmail.com}
\integrante{de Monasterio, Fransisco Jes\'us}{764/13}{XXX@XXX.XXX}
\integrante{Mart\'in Darricades, Mat\'ias Facundo}{480/13}{matiasamd@gmail.com}
\integrante{Solari Saban, Tom\'as Le\'on}{774/14}{tomassolari94@gmail.com}

% Pongan cuantos integrantes quieran

\maketitle


\newpage

% Para crear un indice
\tableofcontents
	

% Forzar salto de pagina
\clearpage


\newpage


\section{Tipos} %Defino los renombres de tipos b�sicos.

\sinonimo{Empleado}{String}
\sinonimo{Energia}{\ent}
\sinonimo{Cantidad}{\ent}
\enum{Bebida}{Pesti Cola, Falsa Naranja, Se ve nada, Agua con Gags, Agua sin Gags}
\enum{Hamburguesa}{McGyver, CukiQueFresco (Cuarto de Kilo con Queso Fresco), McPato, Big Macabra}


\section{Combo} %Defino el tipo combo, con su especificaci�n.

\begin{problema}{nuevoC}{b: Bebida, h: Hamburguesa, d: Energia}{Combo}
\requiere{ d \geq 0 \land d \leq 100} %tambien podria ser \requiere{energiaEnRango(d) }
\asegura{ bebida(\res) == b}
\asegura{ sandwich(\res) == h}
\asegura{ dificultad(\res) == d}
\end{problema}

\begin{problema}{bebidaC}{c: Combo}{Bebida}
\asegura{ \res == bebida(c)}
\end{problema}

\begin{problema}{sandwichC}{c: Combo}{Hamburguesa}
\asegura{ \res == sandwich(c)}
\end{problema}

\begin{problema}{dificultadC}{c: Combo}{Energia}
\asegura{ \res == dificultad(c)}
\end{problema}

 % Aca va la definici�n del tipo.

\begin{problema}{nuevoC}{b: Bebida, h: Hamburguesa, d: Energia}{Combo}
\requiere{ d \geq 0 \land d \leq 100} %tambien podria ser \requiere{energiaEnRango(d) }
\asegura{ bebida(\res) == b}
\asegura{ sandwich(\res) == h}
\asegura{ dificultad(\res) == d}
\end{problema}

\begin{problema}{bebidaC}{c: Combo}{Bebida}
\asegura{ \res == bebida(c)}
\end{problema}

\begin{problema}{sandwichC}{c: Combo}{Hamburguesa}
\asegura{ \res == sandwich(c)}
\end{problema}

\begin{problema}{dificultadC}{c: Combo}{Energia}
\asegura{ \res == dificultad(c)}
\end{problema}

 % La especificaci�n del tipo combo.

\newpage %Salto de p�gina


\section{Pedido} 

\begin{tipo}{Pedido}
	\observador{numero}{p: Pedido}{\ent}
	\observador{atendio}{p: Pedido}{Empleado}
	\observador{combos}{p: Pedido}{[Combo]}
	
	\medskip
	\invariante[numeroPositivo]{numero(p) > 0}
	\invariante[pideAlgo]{|combos(p)| > 0}
\end{tipo}




\begin{tipo}{Pedido}
	\observador{numero}{p: Pedido}{\ent}
	\observador{atendio}{p: Pedido}{Empleado}
	\observador{combos}{p: Pedido}{[Combo]}
	
	\medskip
	\invariante[numeroPositivo]{numero(p) > 0}
	\invariante[pideAlgo]{|combos(p)| > 0}
\end{tipo}




\newpage

\section{Local} 


\begin{problema}{stockBebidasL}{l: Local, b:Bebida}{Cantidad}
\requiere{ bebidaPerteneceLocal(b,l) } 
\asegura{ \res == stockBebidas(l,b)}
\end{problema}

\begin{problema}{stockSandwichesL}{l: Local, h:Hamburguesa}{Cantidad}
\requiere{ sandwichPerteneceLocal(h,l) } 
\asegura{ \res == stockSandwiches(l,h)}
\end{problema}

\begin{problema}{bebidasDelLocalL}{l: Local}{[Bebida]}
\asegura{ mismos (\res , bebidasDelLocal(l))}
\end{problema}
	
\begin{problema}{sandwichesDelLocalL}{l: Local}{[Hamburguesa]}
\asegura{ mismos (\res , sandwichesDelLocal(l))}
\end{problema}

\begin{problema}{empleadosL}{l: Local}{[Empleado]}
\asegura{ mismos (\res , empleados(l))}
\end{problema}

\begin{problema}{desempleadosL}{l: Local}{[Empleado]}
\asegura{ mismos (\res , desempleados(l))}
\end{problema}

\begin{problema}{energiaEmpleadoL}{l: Local, e:Empleado}{Energia}
\requiere{ empleadoPerteneceLocal(e,l) } 
\asegura{ \res == energiaEmpleado(l,e)}
\end{problema}

\begin{problema}{ventasL}{l: Local}{[Pedido]}
\asegura{ mismos (\res , ventas(l))}
\end{problema}

\begin{problema}{unaVentaCadaUno}{l:Local}{\bool}

\asegura{\res == lista}
%quienAtiendeVentasEmpleado(ventas(l),l) me da la lista de empleados actuales que atendieron los pedidos

%problema unaVentaCadaUno (l:Local) = result : Bool
%Indica si las ventas que realizaron los empleados actuales del local l, fueron hechas rotando de manera estricta. Ejemplo:
%Si los empleados actuales del local l son A,B,C, y D es un ex-empleado y se realizaron las ventas B,D,A,D,C,B,D,D,A,C,B,A deber ́ıa devolver verdadero(las ventas de D no deben considerarse ya que corresponde a un ex-empleado).

\end{problema}

\begin{problema}{venderL}{l: Local, p:Pedido}{}
\requiere[numeroPedidoCorrecto]{numero(p) == mayorNumeroPedido(ventas(\pre{l})) + 1}
\requiere[empleadoDelLocal]{empleadoPerteneceLocal(atendio(p),\pre{l})}
\requiere[tieneEnergiaSuficiente]{energiaEmpleado(\pre{l},atendio(p)) \geq energiaPedido(p)} % No estoy seguro de este requiere
\requiere[stockSandwichSuficiente]{(\forall \selector{h}{sandwichDistintos(combos(p))}) ( stockSandwiches(\pre{l}, h) \\ - cuentaSandwich(h,combos(p))) > 0}
\requiere[stockBebidaSuficiente]{(\forall \selector{b}{bebidaDistintos(combos(p))}) ( stockBebidas(\pre{l}, b) \\ - cuentaBebida(b,combos(p))) > 0}
\modifica{l}
\asegura{mismos(bebidasDelLocal(l),bebidasdelLocal(\pre{l}))}
\asegura{mismos(sandwichesDelLocal(l),sandwichesDelLocal(\pre{l}))}
\asegura{mismosEmpleados(empleados(l) \masmas desempleados(l),empleados(\pre{l}) \masmas desempleados(\pre{l}))}
\asegura{(\forall \selector{b}{bebidaDistintos(combos(p))}) (stockBebidas(\pre{l}, b) \\ - cuentaBebida(b,combos(p))) == stockBebidas(l,b)}
\asegura{(\forall \selector{h}{sandwichDistintos(combos(p))}) (stockSandwiches(\pre{l}, h) \\ - cuentaSandwich(h,combos(p))) == stockSandwiches(l,b)}
\asegura{\IfThenElse{(energiaEmpleado(\pre{l},atendio(p)) - energiaPedido(p)) > 0}{ \\(energiaEmpleado(\pre{l},atendio(p)) - energiaPedido(p)) == energiaEmpleado(l,atendio(p))}{ \\mismosEmpleados(desempleados(l),agregarDesempleadoLocal(\pre{l},atendio(p))) \land \\ mismosEmpleados(empleados(l),despedirEmpleadoLocal(\pre{l},atendio(p))) }}
\asegura{\longitud{ventas(l)} == \longitud{ventas(\pre{l})} +1}
\asegura{numero(p) == mayorNumeroPedido(ventas(l))}
\asegura{mismosPedidos(ventas(l), ventas(\pre{l}) \masmas p )}

\end{problema}

\begin{problema}{candidatosAEmpleadosDelMesL}{l: Local}{[Empleado]}
\asegura{mismosEmpleados( \res ,mejoresEmpleados(l))}
\end{problema}


\begin{problema}{sancionL}{l: Local, e:Empleado, n:Energia}{}
\requiere[empleadoDelLocal]{empleadoPerteneceLocal(e,\pre{l})}
\modifica{l}
\asegura{mismos(bebidasDelLocal(l),bebidasdelLocal(\pre{l}))}
\asegura{mismos(sandwichesDelLocal(l),sandwichesDelLocal(\pre{l}))}
\asegura{mismosEmpleados(empleados(l) \masmas desempleados(l),empleados(\pre{l}) \masmas desempleados(\pre{l}))}
\asegura{mismos(ventas(l),ventas(\pre{l}))}
\asegura{(\forall \selector{b}{bebidasDelLocal(\pre{l})})stockBebidas(\pre{l}, b) == stockBebidas(l,b)}
\asegura{(\forall \selector{empleado}{empleados(\pre{l})}, e \neq empleado)energiaEmplado(\pre{l}, empleado) == \\ energiaEmpleado(l,empleado)}
\asegura{(\forall \selector{h}{sandwichesDelLocal(\pre{l})}) stockSandwiches(\pre{l}, h) == stockSandwiches(l,h)}
\asegura{\IfThenElse{(energiaEmpleado(\pre{l},e) - n) > 0}{ (energiaEmpleado(\pre{l},e) - n) == energiaEmpleado(l,e) \\ }{ mismosEmpleados(desempleados(l),agregarDesempleadoLocal(\pre{l},e)) \land \\ mismosEmpleados(empleados(l),despedirEmpleadoLocal(\pre{l},e)) }}
\end{problema}



\begin{problema}{elVagonetaL}{l: Local}{Empleado}
%problema elVagonetaL (l: Local) = result : Empleado
%Devuelve el empleado actual que mas descanso se tom ́o entre pedido y pedido. Por ejemplo, si los empleados son A, B y C, y atendieron los pedidos en el siguiente orden: A-B-C-B-C-B-C-B-C-B-A, el descanso m ́as largo es el de A (9 ventas), porque B solo tiene descansos de una venta, y el descanso m ́as lardo de C es de 2 ventas. En el caso de que un empleado no hubiese atendido pedidos, se toma que su descanso es la cantidad total de ventas que se hicieron en el local. En el caso de que un empleado hubiese atendido al menos un pedido, tener en cuenta que tambi ́en se contabilizan como descansos a) la cantidad de pedidos entre el comienzo y el primer pedido que atendio, y b) entre el ultimo pedido que atendio y el total de pedidos.
\end{problema}

\begin{problema}{anularPedidoL}{l: Local, n: \ent}{}
\requiere{esPedidoLocal(\pre{l},n)}
\requiere[empleadoDelLocal]{empleadoPerteneceLocal(atendio(pedidoLocal(\pre{l},n)),\pre{l})}
\modifica{l}
\asegura{mismos(bebidasDelLocal(l),bebidasdelLocal(\pre{l}))}
\asegura{mismos(sandwichesDelLocal(l),sandwichesDelLocal(\pre{l}))}
\asegura{mismosEmpleados(empleados(l),empleados(\pre{l}))}
\asegura{mismosEmpleados(desempleados(l),desempleados(\pre{l}))}
\asegura{\longitud{ventas(l)} == \longitud{ventas(\pre{l})} - 1}
\asegura{(\forall \selector{b}{bebidaDistintos(combos(pedidoLocal(\pre{l},n)))}) (stockBebidas(\pre{l}, b) \\ + cuentaBebida(b,combos(pedidoLocal(\pre{l},n)))) == stockBebidas(l,b)}
\asegura{(\forall \selector{h}{sandwichDistintos(combos(pedidoLocal(\pre{l},n)))}) (stockSandwiches(\pre{l}, h) \\ - cuentaSandwich(h,combos(pedidoLocal(\pre{l},n)))) == stockSandwiches(l,b)}
\asegura{(\forall \selector{e}{empleados(\pre{l})}, atendio(pedidoLocal(\pre{l},n)) \neq e)energiaEmplado(\pre{l}, e) == \\ energiaEmpleado(l,e)}
\asegura{ (energiaEmpleado(\pre{l},atendio(pedidoLocal(\pre{l},n))) + energiaPedido(pedidoLocal(\pre{l},n))) == energiaEmpleado(l,atendio(pedidoLocal(\pre{l},n)))
}
\asegura{mayorNumeroPedido(ventas(\pre{l})) - 1 == mayorNumeroPedido(ventas(l))}
\asegura{mismosPedidosNoNumero(ventas(l), eliminarPedidoVentasNumero(\pre{l}, n))}
\asegura{(\forall \selector{i}{\longitud{ventas(\pre{l})}}, numero(ventas(\pre{l})_{i}) \neq n ) \IfThenElse{ numero(ventas(\pre{l})_{i}) > n}{ \\ numero(ventas(\pre{l})_{i}) - 1 ==numero(ventas(l)_{i})}{numero(ventas(\pre{l})_{i}) ==numero(ventas(l)_{i}) } }

\end{problema}

\begin{problema}{agregarComboAlPedidoL}{l: Local, c: Combo, n:\ent}{}
\requiere{esPedidoLocal(\pre{l},n)}
\requiere[empleadoDelLocal]{empleadoPerteneceLocal(atendio(pedidoLocal(\pre{l},n)),\pre{l})}
\requiere[tieneEnergiaSuficiente]{energiaEmpleado(\pre{l},atendio(pedidoLocal(\pre{l},n)))) \geq \\ dificultad(c)}
\requiere[stockSandwichSuficiente]{(stockSandwiches(\pre{l}, sandwich(c)) - 1) > 0}
\requiere[stockBebidaSuficiente]{( stockBebidas(\pre{l}, bebida(c)) - 1) > 0}
\modifica{l}
\asegura{mismos(bebidasDelLocal(l),bebidasdelLocal(\pre{l}))}
\asegura{mismos(sandwichesDelLocal(l),sandwichesDelLocal(\pre{l}))}
\asegura{mismosEmpleados(empleados(l),empleados(\pre{l}))}
\asegura{mismosEmpleados(desempleados(l),desempleados(\pre{l}))}
\asegura{(\forall \selector{b}{bebidasDelLocal(\pre{l})}, b \neq bebida(c))stockBebidas(\pre{l}, b) == stockBebidas(l,b)}
\asegura{(\forall \selector{e}{empleados(\pre{l})}, e \neq atendio((pedidoLocal(\pre{l},n)))energiaEmplado(\pre{l}, e) == \\ energiaEmpleado(l,e)}
\asegura{(\forall \selector{h}{sandwichesDelLocal(\pre{l})}, h \neq sandwich(c)) stockSandwiches(\pre{l}, h) \\ == stockSandwiches(l,h)}
\asegura{stockBebidas(\pre{l}, bebida(c)) == stockBebidas(l,bebida(c)) + 1}
\asegura{stockSandwiches(\pre{l}, sandwich(c)) == stockSandwiches(l,sandwich(c)) + 1}
\asegura{energiaEmplado(\pre{l}, atendio(pedidoLocal(\pre{l},n))) == \\ energiaEmpleado(l,atendio(pedidoLocal(\pre{l},n))) + dificultad(c)}
\asegura{\longitud{ventas(l)} == \longitud{ventas(\pre{l})}}
\asegura{mismosPedidos( eliminarPedidoVentasNumero(\pre{l}, n), eliminarPedidoVentasNumero(l, n)) }
\asegura{ numero(pedidoLocal(\pre{l},n)) == numero(pedidoLocal(l,n))}
\asegura{ atendio(pedidoLocal(\pre{l},n)) == atendio(pedidoLocal(l,n))}
\asegura{ mismosCombosDePedidos(combos(pedidoLocal(\pre{l},n))++[c],combos(pedidoLocal(l,n)))}

 \end{problema}


\begin{problema}{stockBebidasL}{l: Local, b:Bebida}{Cantidad}
\requiere{ bebidaPerteneceLocal(b,l) } 
\asegura{ \res == stockBebidas(l,b)}
\end{problema}

\begin{problema}{stockSandwichesL}{l: Local, h:Hamburguesa}{Cantidad}
\requiere{ sandwichPerteneceLocal(h,l) } 
\asegura{ \res == stockSandwiches(l,h)}
\end{problema}

\begin{problema}{bebidasDelLocalL}{l: Local}{[Bebida]}
\asegura{ mismos (\res , bebidasDelLocal(l))}
\end{problema}
	
\begin{problema}{sandwichesDelLocalL}{l: Local}{[Hamburguesa]}
\asegura{ mismos (\res , sandwichesDelLocal(l))}
\end{problema}

\begin{problema}{empleadosL}{l: Local}{[Empleado]}
\asegura{ mismos (\res , empleados(l))}
\end{problema}

\begin{problema}{desempleadosL}{l: Local}{[Empleado]}
\asegura{ mismos (\res , desempleados(l))}
\end{problema}

\begin{problema}{energiaEmpleadoL}{l: Local, e:Empleado}{Energia}
\requiere{ empleadoPerteneceLocal(e,l) } 
\asegura{ \res == energiaEmpleado(l,e)}
\end{problema}

\begin{problema}{ventasL}{l: Local}{[Pedido]}
\asegura{ mismos (\res , ventas(l))}
\end{problema}

\begin{problema}{unaVentaCadaUno}{l:Local}{\bool}

\asegura{\res == lista}
%quienAtiendeVentasEmpleado(ventas(l),l) me da la lista de empleados actuales que atendieron los pedidos

%problema unaVentaCadaUno (l:Local) = result : Bool
%Indica si las ventas que realizaron los empleados actuales del local l, fueron hechas rotando de manera estricta. Ejemplo:
%Si los empleados actuales del local l son A,B,C, y D es un ex-empleado y se realizaron las ventas B,D,A,D,C,B,D,D,A,C,B,A deber ́ıa devolver verdadero(las ventas de D no deben considerarse ya que corresponde a un ex-empleado).

\end{problema}

\begin{problema}{venderL}{l: Local, p:Pedido}{}
\requiere[numeroPedidoCorrecto]{numero(p) == mayorNumeroPedido(ventas(\pre{l})) + 1}
\requiere[empleadoDelLocal]{empleadoPerteneceLocal(atendio(p),\pre{l})}
\requiere[tieneEnergiaSuficiente]{energiaEmpleado(\pre{l},atendio(p)) \geq energiaPedido(p)} % No estoy seguro de este requiere
\requiere[stockSandwichSuficiente]{(\forall \selector{h}{sandwichDistintos(combos(p))}) ( stockSandwiches(\pre{l}, h) \\ - cuentaSandwich(h,combos(p))) > 0}
\requiere[stockBebidaSuficiente]{(\forall \selector{b}{bebidaDistintos(combos(p))}) ( stockBebidas(\pre{l}, b) \\ - cuentaBebida(b,combos(p))) > 0}
\modifica{l}
\asegura{mismos(bebidasDelLocal(l),bebidasdelLocal(\pre{l}))}
\asegura{mismos(sandwichesDelLocal(l),sandwichesDelLocal(\pre{l}))}
\asegura{mismosEmpleados(empleados(l) \masmas desempleados(l),empleados(\pre{l}) \masmas desempleados(\pre{l}))}
\asegura{(\forall \selector{b}{bebidaDistintos(combos(p))}) (stockBebidas(\pre{l}, b) \\ - cuentaBebida(b,combos(p))) == stockBebidas(l,b)}
\asegura{(\forall \selector{h}{sandwichDistintos(combos(p))}) (stockSandwiches(\pre{l}, h) \\ - cuentaSandwich(h,combos(p))) == stockSandwiches(l,b)}
\asegura{\IfThenElse{(energiaEmpleado(\pre{l},atendio(p)) - energiaPedido(p)) > 0}{ \\(energiaEmpleado(\pre{l},atendio(p)) - energiaPedido(p)) == energiaEmpleado(l,atendio(p))}{ \\mismosEmpleados(desempleados(l),agregarDesempleadoLocal(\pre{l},atendio(p))) \land \\ mismosEmpleados(empleados(l),despedirEmpleadoLocal(\pre{l},atendio(p))) }}
\asegura{\longitud{ventas(l)} == \longitud{ventas(\pre{l})} +1}
\asegura{numero(p) == mayorNumeroPedido(ventas(l))}
\asegura{mismosPedidos(ventas(l), ventas(\pre{l}) \masmas p )}

\end{problema}

\begin{problema}{candidatosAEmpleadosDelMesL}{l: Local}{[Empleado]}
\asegura{mismosEmpleados( \res ,mejoresEmpleados(l))}
\end{problema}


\begin{problema}{sancionL}{l: Local, e:Empleado, n:Energia}{}
\requiere[empleadoDelLocal]{empleadoPerteneceLocal(e,\pre{l})}
\modifica{l}
\asegura{mismos(bebidasDelLocal(l),bebidasdelLocal(\pre{l}))}
\asegura{mismos(sandwichesDelLocal(l),sandwichesDelLocal(\pre{l}))}
\asegura{mismosEmpleados(empleados(l) \masmas desempleados(l),empleados(\pre{l}) \masmas desempleados(\pre{l}))}
\asegura{mismos(ventas(l),ventas(\pre{l}))}
\asegura{(\forall \selector{b}{bebidasDelLocal(\pre{l})})stockBebidas(\pre{l}, b) == stockBebidas(l,b)}
\asegura{(\forall \selector{empleado}{empleados(\pre{l})}, e \neq empleado)energiaEmplado(\pre{l}, empleado) == \\ energiaEmpleado(l,empleado)}
\asegura{(\forall \selector{h}{sandwichesDelLocal(\pre{l})}) stockSandwiches(\pre{l}, h) == stockSandwiches(l,h)}
\asegura{\IfThenElse{(energiaEmpleado(\pre{l},e) - n) > 0}{ (energiaEmpleado(\pre{l},e) - n) == energiaEmpleado(l,e) \\ }{ mismosEmpleados(desempleados(l),agregarDesempleadoLocal(\pre{l},e)) \land \\ mismosEmpleados(empleados(l),despedirEmpleadoLocal(\pre{l},e)) }}
\end{problema}



\begin{problema}{elVagonetaL}{l: Local}{Empleado}
%problema elVagonetaL (l: Local) = result : Empleado
%Devuelve el empleado actual que mas descanso se tom ́o entre pedido y pedido. Por ejemplo, si los empleados son A, B y C, y atendieron los pedidos en el siguiente orden: A-B-C-B-C-B-C-B-C-B-A, el descanso m ́as largo es el de A (9 ventas), porque B solo tiene descansos de una venta, y el descanso m ́as lardo de C es de 2 ventas. En el caso de que un empleado no hubiese atendido pedidos, se toma que su descanso es la cantidad total de ventas que se hicieron en el local. En el caso de que un empleado hubiese atendido al menos un pedido, tener en cuenta que tambi ́en se contabilizan como descansos a) la cantidad de pedidos entre el comienzo y el primer pedido que atendio, y b) entre el ultimo pedido que atendio y el total de pedidos.
\end{problema}

\begin{problema}{anularPedidoL}{l: Local, n: \ent}{}
\requiere{esPedidoLocal(\pre{l},n)}
\requiere[empleadoDelLocal]{empleadoPerteneceLocal(atendio(pedidoLocal(\pre{l},n)),\pre{l})}
\modifica{l}
\asegura{mismos(bebidasDelLocal(l),bebidasdelLocal(\pre{l}))}
\asegura{mismos(sandwichesDelLocal(l),sandwichesDelLocal(\pre{l}))}
\asegura{mismosEmpleados(empleados(l),empleados(\pre{l}))}
\asegura{mismosEmpleados(desempleados(l),desempleados(\pre{l}))}
\asegura{\longitud{ventas(l)} == \longitud{ventas(\pre{l})} - 1}
\asegura{(\forall \selector{b}{bebidaDistintos(combos(pedidoLocal(\pre{l},n)))}) (stockBebidas(\pre{l}, b) \\ + cuentaBebida(b,combos(pedidoLocal(\pre{l},n)))) == stockBebidas(l,b)}
\asegura{(\forall \selector{h}{sandwichDistintos(combos(pedidoLocal(\pre{l},n)))}) (stockSandwiches(\pre{l}, h) \\ - cuentaSandwich(h,combos(pedidoLocal(\pre{l},n)))) == stockSandwiches(l,b)}
\asegura{(\forall \selector{e}{empleados(\pre{l})}, atendio(pedidoLocal(\pre{l},n)) \neq e)energiaEmplado(\pre{l}, e) == \\ energiaEmpleado(l,e)}
\asegura{ (energiaEmpleado(\pre{l},atendio(pedidoLocal(\pre{l},n))) + energiaPedido(pedidoLocal(\pre{l},n))) == energiaEmpleado(l,atendio(pedidoLocal(\pre{l},n)))
}
\asegura{mayorNumeroPedido(ventas(\pre{l})) - 1 == mayorNumeroPedido(ventas(l))}
\asegura{mismosPedidosNoNumero(ventas(l), eliminarPedidoVentasNumero(\pre{l}, n))}
\asegura{(\forall \selector{i}{\longitud{ventas(\pre{l})}}, numero(ventas(\pre{l})_{i}) \neq n ) \IfThenElse{ numero(ventas(\pre{l})_{i}) > n}{ \\ numero(ventas(\pre{l})_{i}) - 1 ==numero(ventas(l)_{i})}{numero(ventas(\pre{l})_{i}) ==numero(ventas(l)_{i}) } }

\end{problema}

\begin{problema}{agregarComboAlPedidoL}{l: Local, c: Combo, n:\ent}{}
\requiere{esPedidoLocal(\pre{l},n)}
\requiere[empleadoDelLocal]{empleadoPerteneceLocal(atendio(pedidoLocal(\pre{l},n)),\pre{l})}
\requiere[tieneEnergiaSuficiente]{energiaEmpleado(\pre{l},atendio(pedidoLocal(\pre{l},n)))) \geq \\ dificultad(c)}
\requiere[stockSandwichSuficiente]{(stockSandwiches(\pre{l}, sandwich(c)) - 1) > 0}
\requiere[stockBebidaSuficiente]{( stockBebidas(\pre{l}, bebida(c)) - 1) > 0}
\modifica{l}
\asegura{mismos(bebidasDelLocal(l),bebidasdelLocal(\pre{l}))}
\asegura{mismos(sandwichesDelLocal(l),sandwichesDelLocal(\pre{l}))}
\asegura{mismosEmpleados(empleados(l),empleados(\pre{l}))}
\asegura{mismosEmpleados(desempleados(l),desempleados(\pre{l}))}
\asegura{(\forall \selector{b}{bebidasDelLocal(\pre{l})}, b \neq bebida(c))stockBebidas(\pre{l}, b) == stockBebidas(l,b)}
\asegura{(\forall \selector{e}{empleados(\pre{l})}, e \neq atendio((pedidoLocal(\pre{l},n)))energiaEmplado(\pre{l}, e) == \\ energiaEmpleado(l,e)}
\asegura{(\forall \selector{h}{sandwichesDelLocal(\pre{l})}, h \neq sandwich(c)) stockSandwiches(\pre{l}, h) \\ == stockSandwiches(l,h)}
\asegura{stockBebidas(\pre{l}, bebida(c)) == stockBebidas(l,bebida(c)) + 1}
\asegura{stockSandwiches(\pre{l}, sandwich(c)) == stockSandwiches(l,sandwich(c)) + 1}
\asegura{energiaEmplado(\pre{l}, atendio(pedidoLocal(\pre{l},n))) == \\ energiaEmpleado(l,atendio(pedidoLocal(\pre{l},n))) + dificultad(c)}
\asegura{\longitud{ventas(l)} == \longitud{ventas(\pre{l})}}
\asegura{mismosPedidos( eliminarPedidoVentasNumero(\pre{l}, n), eliminarPedidoVentasNumero(l, n)) }
\asegura{ numero(pedidoLocal(\pre{l},n)) == numero(pedidoLocal(l,n))}
\asegura{ atendio(pedidoLocal(\pre{l},n)) == atendio(pedidoLocal(l,n))}
\asegura{ mismosCombosDePedidos(combos(pedidoLocal(\pre{l},n))++[c],combos(pedidoLocal(l,n)))}

 \end{problema}

\newpage

\section{Funciones Auxiliares} 

\aux{distintos}{ls:[T]}{\bool}{ 
  (\forall i,j \selec [0..|ls|), i \neq j) ls_i \neq ls_j
}


\aux{energiaEnRango}{e: Energia} {\bool}{
        0 \leq e \leq 100
}

\subsection{Combo}
% los aux del tipo combo

\subsection{Pedido}
\aux{combosIguales}{cA: Combo, cB: Combo}{\bool}{ 
  bebida(cA) == bebida(cB) \land sandwich(cA) == sandwich(cB)% \land dificultad(cA) == dificultad(cB) Segun el mail que mandaron la parte de dificultad no seria necesaria ya que cuando aclararon el problema elMezcladito dijeron "Si tienen distinta dificultad no se consideran distintos combos"
}
\aux{cantidadSandwichDistintos}{combos: \TLista{Combo}}{\ent}{
|[ combos_{i} | \selector{i}{\rangoca{0}{\longitud{combos}}}, \forall \selector{j}{\rangoca{0}{\longitud{combos}}}, i \leq j \land \IfThenElse{i < j}{sandwich(combos_{i}) \neq sandwich(combos_{j})}{\True} ]|
}
\aux{cantidadBebidaDistintos}{combos: \TLista{Combo}}{\ent}{
|[ combos_{i} | \selector{i}{\rangoca{0}{\longitud{combos}}}, \forall \selector{j}{\rangoca{0}{\longitud{combos}}}, i \leq j \land \IfThenElse{i < j}{bebida(combos_{i}) \neq bebida(combos_{j})}{\True} ]|
}
\aux{combosDistintos}{cA: Combo, cB: Combo}{\bool}{ 
  bebida(cA) \neq bebida(cB) \lor sandwich(cA) \neq sandwich(cB)% \land dificultad(cA) == dificultad(cB) Segun el mail que mandaron la parte de dificultad no seria necesaria ya que cuando aclararon el problema elMezcladito dijeron "Si tienen distinta dificultad no se consideran distintos combos"
}
\aux{sandwichDistintos}{combos: \TLista{Combo}}{\TLista{Hamburguesa}}{
[ sandwich(combos_{i}) | \selector{i}{\rangoca{0}{\longitud{combos}}}, \\ \forall \selector{j}{\rangoca{0}{\longitud{combos}}}, i \leq j \land \IfThenElse{i < j}{sandwich(combos_{i}) \neq sandwich(combos_{j})}{\True} ]
}
\aux{bebidaDistintos}{combos: \TLista{Combo}}{\TLista{Bebida}}{
[ bebida(combos_{i}) | \selector{i}{\rangoca{0}{\longitud{combos}}}, \forall \selector{j}{\rangoca{0}{\longitud{combos}}}, i \leq j \land \IfThenElse{i < j}{bebida(combos_{i}) \neq bebida(combos_{j})}{\True} ]
}

\aux{mismosSandwich}{combosA: \TLista{Combo}, combosB: \TLista{Combo}}{\bool}{
\longitud{sandwichDistintos(combosA)} == \\ \longitud{sandwichDistintos(combosB)} \land (\forall \selector{sandwichA}{sandwichDistintos(combosA)},sandwichB \selec \\ sandwichDistintos(combosB)) sandwichA == sandwichB
}
\aux{mismosBebida}{combosA: \TLista{Combo}, combosB: \TLista{Combo}}{\bool}{
\longitud{bebidaDistintos(combosA)} == \\ \longitud{bebidaDistintos(combosB)} \land (\forall \selector{bebidaA}{bebidaDistintos(combosA)}, \selector{bebidaB}{bebidaDistintos(combosB)}) \\ bebidaA == bebidaB
}

% los aux del tipo pedido

\subsection{Local}

\aux{menorNumeroPedido}{ventas: \TLista{Pedido}}{\ent}{
\IfThenElse{/longitud{ventas} > 1}{|[ numero(ventas_{i}) | \selector{i}{\rangoca{0}{\longitud{ventas}}},\\ \forall \selector{j}{\rangoca{0}{\longitud{ventas}}}, i \neq j \land numero(ventas_{i}) < numero(ventas_{j}) ]|}{numero(ventas_{0})}
}

\aux{mayorNumeroPedido}{ventas: \TLista{Pedido}}{\ent}{
\IfThenElse{/longitud{ventas} > 1}{|[ numero(ventas_{i}) | \selector{i}{\rangoca{0}{\longitud{ventas}}},\\ \forall \selector{j}{\rangoca{0}{\longitud{ventas}}}, i \neq j \land numero(ventas_{i}) > numero(ventas_{j}) ]|}{numero(ventas_{0})}
}

\aux{sandwichLocal}{ventas: \TLista{Pedido}}{\TLista{Hamburguesa}}{
[ sandwichDistintos(combos(ventas_{i}))_{j} | \selector{i}{\rangoca{0}{\longitud{ventas}}}, \\ \forall \selector{j}{\rangoca{0}{\longitud{sandwichDistintos(combos(ventas_{i}))}}}]
}

\aux{bebidaLocal}{ventas: \TLista{Pedido}}{\TLista{Bebida}}{
[ bebidaDistintos(combos(ventas_{i}))_{j} | \selector{i}{\rangoca{0}{\longitud{ventas}}}, \\ \forall \selector{j}{\rangoca{0}{\longitud{bebidaDistintos(combos(ventas_{i}))}}}]
}

\aux{quienAtiendeVentasEmpleado}{ventas: \TLista{Pedido}, l: Local}{\TLista{Empleado}}{
[atendio(ventas_{i}) | \selector{i}{\rangoca{0}{\longitud{ventas}}}, \\ empleadoPerteneceLocal(atendio(ventas_{i}), l)]
}
\aux{empleadoPerteneceLocal}{e: Empleado, l: Local}{\bool}{
(\exists \selector{i}{empleados(l)}) e == i
}
\aux{bebidaPerteneceLocal}{b: Bebida, l: Local}{\bool}{
(\exists \selector{i}{bebidasDelLocal(l)}) e == i
}
\aux{sandwichPerteneceLocal}{h: Hamburguesa, l: Local}{\bool}{
(\exists \selector{i}{sandwichesDelLocal(l)}) e == i
}

\aux{energiaPedido}{p: Pedido}{\ent}{
\sum [ dificultad(combos(p)_{i}) | \selector{i}{\rangoca{0}{\longitud{combos(p)}}}]
}

\aux{cuentaSandwich}{h: Hamburgesa, combos: \TLista{Combo}}{\ent}{
|[sandwich(c)| \selector{c}{combos}, sandwich(c) == h  ]|
}

\aux{cuentaBebida}{b: Bebida, combos: \TLista{Combo}}{\ent}{
|[bebida(c)| \selector{c}{combos}, bebida(c) == b  ]|
}

\aux{despedirEmpleadoLocal}{l:Local,e: Empleado}{\TLista{Empleado}}{
[empleados(l)_{i} | \selector{i}{\longitud{empleados(l)}},\\ empleados(l)_{i} \neq e ]
}

\aux{agregarDesempleadoLocal}{l:Local,e: Empleado}{\TLista{Empleado}}{
\TLista{desempleados(l)} \masmas \TLista{e} 
}

\aux{mismosEmpleados}{empleadosA: \TLista{Empleado},empleadosB: \TLista{Empleado}}{\bool}{
\longitud{empleadosA} == \\ \longitud{empleadosB)} \land (\forall \selector{empleadoA}{empleadosA}, \selector{empleadoB}{empleadosB})  empleadoA == empleadoB
}



% los aux del tipo local






\end{document} %Termin�!


%Muestra de como poner las cosas para el TPI

%\section{Implementaci\'on}




%\subsection{Flor}

%\lstinputlisting[caption=Flor.h]{../src/Flor.h}

%\lstinputlisting[caption=Flor.cpp]{../src/Flor.cpp}

%\newpage



